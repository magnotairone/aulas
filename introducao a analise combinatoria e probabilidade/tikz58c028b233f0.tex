\documentclass{article}
\include{preview}
\usepackage[pdftex,active,tightpage]{preview}
\usepackage{amsmath}
\usepackage{tikz}
\usetikzlibrary{matrix}

%% INSERT YOUR OWN CODE HERE 
\begin{document}
\begin{preview}
%% TIKZ_CODE %%
\begin{center}

\tikzset{every picture/.style={line width=0.75pt}} %set default line width to 0.75pt        

\begin{tikzpicture}[x=0.75pt,y=0.75pt,yscale=-1,xscale=1]
%uncomment if require: \path (0,300); %set diagram left start at 0, and has height of 300

%Shape: Circle [id:dp2436306557760699] 
\draw   (100,99.5) .. controls (100,63.33) and (129.33,34) .. (165.5,34) .. controls (201.67,34) and (231,63.33) .. (231,99.5) .. controls (231,135.67) and (201.67,165) .. (165.5,165) .. controls (129.33,165) and (100,135.67) .. (100,99.5) -- cycle ;
%Shape: Circle [id:dp9832570701717033] 
\draw   (180,100) .. controls (180,64.1) and (209.1,35) .. (245,35) .. controls (280.9,35) and (310,64.1) .. (310,100) .. controls (310,135.9) and (280.9,165) .. (245,165) .. controls (209.1,165) and (180,135.9) .. (180,100) -- cycle ;
%Shape: Rectangle [id:dp7045960480023841] 
\draw   (46,12) -- (372,12) -- (372,189) -- (46,189) -- cycle ;

% Text Node
\draw (132,87) node [anchor=north west][inner sep=0.75pt]   [align=left] {5};
% Text Node
\draw (59,162) node [anchor=north west][inner sep=0.75pt]   [align=left] {$\displaystyle \Omega $};
% Text Node
\draw (110,31) node [anchor=north west][inner sep=0.75pt]   [align=left] {A};
% Text Node
\draw (287,32) node [anchor=north west][inner sep=0.75pt]   [align=left] {B};
% Text Node
\draw (259,84) node [anchor=north west][inner sep=0.75pt]   [align=left] {2};
% Text Node
\draw (199,79) node [anchor=north west][inner sep=0.75pt]   [align=left] {1\\3};

\end{tikzpicture}
\end{center}

\end{preview}
\end{document}
